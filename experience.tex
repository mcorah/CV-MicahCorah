\cvsection{Experience}

\begin{cventries}
  \cventrytwo
  {Carnegie Mellon University}
  {Pittsburgh, PA}
  {Research Assistant}
  {Aug 2015--Present}
  {
    \begin{cvitems} % Description(s) bullet points
    \item Advisor: \emph{Prof. Nathan Michael}
    \item Developed algorithms and analysis techniques for multi-robot sensing,
      coverage, exploration, and target tracking based on submodular
      maximization, higher-order monotonicity conditions, and spatial locality
    \item Design and analysis of a planner for exploration at high speed (2.25
      m/s) with an aerial robot in collaboration with Kshitij Goel and Curtis
      Boirum.
      This system was tested in simulation and on a hexrotor robot, outdoors, on
      the CMU campus
    \item Developed a system for multi-robot exploration combining Cauchy-Schwarz
      mutual information for ranging sensors, Monte-Carlo tree search for path
      path planning, and multi-robot planning via submodular maximization
    \item Implemented core components of a system providing control and autonomy
      for aerial robots.
      Contributions include trajectory representation and management and a
      modular finite state machine
    \end{cvitems}
  }
  {Teaching Assistant}
  {Aug--Dec 2018}
  {
    \begin{cvitems} % Description(s) bullet points
    \item Instructor: \emph{Prof. Michael Erdmann}
    \item Course: Mathematical Fundamentals for Robotics (16-811)
    \item Responsibilities: grading assignments, holding office hours
    \item Prepared and gave a lecture on submodular maximization
    \end{cvitems}
  }

  \cventrytwo
  {Carnegie Mellon University (Internships)}
  {Pittsburgh, PA}
  {Research Intern: Persistent Coverage}
  {May--Aug 2014}
  {
    \begin{cvitems}
    \item Advisor: \emph{Prof. Nathan Michael}
    \item NSF Research Experience for Undergraduates (REU)
    \item Implemented minimum snap, collision free, multi-vehicle trajectory
      generation
    \item Implemented controller for tracking of discretized trajectories
    %\item Developed system for persistent flight, recharging, and coverage of waypoints
    %\item Interview: \url{https://www.youtube.com/watch?v=qqpLp2Ayqcw}
    \end{cvitems}
  }
  {Research Intern: Wing Assembly}
  {May--Aug 2013}
  {
    \begin{cvitems}
    \item Advisor: \emph{Prof. Reid Simmons}
    \item Developed a simulation of multi-robot assembly of an airplane wing-ladder
    \item Implemented an autonomous behavior where a mobile robot attaches and
      aligns to an airplane wing spar
    \end{cvitems}
  }

  \cventrytwo
  {Rensselaer Polytechnic Institute}
  {Troy, NY}
  {Independent Study: Robotic Catching}
  {Aug--Dec 2013}
  {
    \begin{cvitems}
    \item Advisor: \emph{Prof. Jeff Trinkle}
    \item Modeling and simulation of contact-oriented catching of a sliding object
    \end{cvitems}
  }
  {Undergraduate Researcher: Scientific Computing}
  {Sept 2012--Dec 2013}
  {
    \begin{cvitems}
    \item Implemented threaded mesh I/O for the Parallel Unstructured Mesh Interface
    %\item Transitioned software from partition-model-per-part to partition-model-per-process
    \end{cvitems}
  }

  \cventry
  {}
  {}
  {Undergraduate Teaching Assistant (Computer Science 1)}
  {Aug 2012--May 2013}
  {
    \begin{cvitems}
    \item Assisted students with lab work and graded results
    \end{cvitems}
  }
\end{cventries}
