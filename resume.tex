% LaTeX file for resume 
% This file uses the resume document class (res.cls)

%\documentclass[margin,letterpaper]{res}
\documentclass[margin]{res}
%\usepackage{helvetica} % uses helvetica postscript font (download helvetica.sty)
\usepackage{hyperref}
%{-1.5mm}\usepackage{newcent}   % uses new century schoolbook postscript font  
\topmargin=-.75in  % start text higher on the page
\footskip=-0.5in
\hoffset=0.5in
\textwidth=6.5in
\setlength{\textheight}{10.5in} % increase text height to fit resume on 1 page
\begin{document}  
\name{Micah D. Corah}

\address{\bf Current\\ 42 Colvin Circle  \\   Troy, NY 12180  \\ {\bf phone: }(720) 270-3901 \\ {\bf email: }micahcorah@gmail.com}
\address{\bf Permanent\\ 3361 E. Maplewood Ave. \\   Centennial, CO 80121}

\begin{resume}                        
\vspace{-1.5mm}
\section{SUMMARY OF SKILLS:}
\vspace{10mm}
  \begin{itemize}
    \item Robotics simulation, programming, and design
    \item Parallel and high performance computing for scientific computing
    \item Experienced with CAD (NX and Solidworks) with a focus on surfacing techniques
    \item Prototyping: 3d printing, CNC milling, vacuum forming
    \item Languages used: C, C++, Python, Matlab, Bash, \LaTeX, Task Description Language
	\end{itemize}
\vspace{-3.5mm}
\section{EDUCATION:}
	Rensselaer Polytechnic Institute, Troy, NY \\
	Bachelor's: Computer Science and Mechanical Engineering, May 2015 \\
	G.P.A. 3.97/4.00, Dean's List
\vspace{-2.5mm}
\normalsize{\section{Summary of Relevant Coursework:}}
   \begin{ncolumn}{2}
		{\bf Computer Science}		  	      		&  {\bf Engineering} \\
	\vspace{-1.5mm}
      Intro. to Discrete Structures &	 Strengths of Materials \\
			Intro. to Algorithms				  &  Intro. to Engineering Design \\  
			Models of Computation				  &  Embedded Control \\
      Computer Organization          &  Elements of Mechanical Design \\
			Parallel Programming				  &  Electronic Instrumentation \\
			Operating Systems						  &  Robotics 1 \\
      
	\end{ncolumn}
\vspace{-5mm}
 \section{EXPERIENCE:}      
	\begin{tabular}{p{4in} r}
    {\bf Independent Study: Robotic Catching}, Jeff Trinkle, RPI & August 2013 -- Present
  \end{tabular}
  \begin{itemize}
    \item Developing a simulation and controller for contact-centric catching of a sliding object
    \item Matlab, C++
  \end{itemize}
	\vspace{-1.5mm}
	\begin{tabular}{p{4in} r}
    {\bf Internship: Multi-Robot Assembly}, Reid Simmons, CMU & May 2013 -- August 2013
  \end{tabular}
  \begin{itemize}
    \item Developed a simulation of multi-robot assembly of an airplane wing-ladder
    \item Implemented a controller for latching a robot onto the wing 
    \item C++, Task Description Language
  \end{itemize}
	\vspace{-1.5mm}
	\begin{tabular}{p{4in} r}
		{\bf Research: High Performance Computing}, SCOREC, RPI	& September 2012 -- May 2013\\
	\end{tabular}
	\begin{itemize}
		\item Implemented threaded mesh I/O for the Parallel Unstructured Mesh Interface
    \item Transitioned software from partition-model-per-part to partition-model-per-process
		\item Debugged and tested FMDB/PUMI on the BlueGene/Q and Intel Phi
    \item C++
	\end{itemize}
	\vspace{-1.5mm}
	\begin{tabular}{p{4in} r}
	  {\bf Undergraduate TA}, Computer Science 1, RPI 			& August 2012 -- May 2013\\
	\end{tabular}
	\begin{itemize}
		\item Assisted students with lab-work and graded results
 	\end{itemize}
	\vspace{-1.5mm}
	\begin{tabular}{p{4in} r}
		{\bf Research}, Cognitive Robotics Lab, RPI		& September 2012 -- December 2012
	\end{tabular}
	\begin{itemize}
		\item Development of server for Robot Operating System based actor
			controlled by Unity tablet application.
	\end{itemize}
\vspace{-1.5mm}

\normalsize{\section{Assorted Projects}}
\begin{itemize}
  \item Developed sleep-monitor which turns off electronics when user falls asleep (Intro. to Engr. Design)
  \item Designed and produced a longboard press and several longboards using NX CAD, including a longboard
        with interior LEDs
  \item Designed and produced various themed CO2 cars including a racing pickup, Tron Light-Cycle,
        and autonomous golf cart, utilizing Solidworks, 3d printing, and CNC milling
\end{itemize}
\vspace{-1.5mm}

\section{ACTIVITIES:}
	\begin{itemize}
    \item RPI Sample Return Challenge Team
    \begin{itemize}
      \item Competition involves autonomously finding and retrieving objects in an outdoor environment
      \item Involved in mechanical design, part selection, and eventually programming in preparation for the summer 2014 competition
    \end{itemize}
	\end{itemize}
\end{resume} 
\end{document}


