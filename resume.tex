% LaTeX file for resume 
% This file uses the resume document class (res.cls)
%\documentclass[margin,letterpaper]{res}
\documentclass[margin]{res}
\usepackage{hyperref}

\topmargin=-.5in  % start text higher on the page
\footskip=-0.5in
\hoffset=0.5in
\textwidth=6.5in
\setlength{\textheight}{10.2in} % increase text height to fit resume on 1 page

\newcommand{\experience}[4]{
	\begin{tabular}{p{4in} r}
    {\bf #1} #2 & #3
  \end{tabular}
  \begin{itemize}
      #4
  \end{itemize}
}

\begin{document}  
\name{Micah David Corah}

\address{\bf Permanent\\ 3361 E. Maplewood Ave. \\   Centennial, CO 80121\\ {\bf phone: }(720) 270-3901 \\ {\bf email: }micahcorah@gmail.com}
\address{\bf Current\\ Williams 302c \\ 1999 Burdett Ave. \\   Troy, NY 12180  }

\begin{resume}                        
\section{SUMMARY OF SKILLS:}
  \begin{itemize}
    \item Simulation of multi-robot systems
    \item Software development for robotics especially ROS
    \item Control and trajectory generation, especially regarding quadrotors
    \item Parallel high performance and scientific computing (MPI)
    \item Engineering design especially using CAD (Solidworks and NX)
    \item Prototyping: 3d printing, CNC milling, vacuum forming
    \item Languages: C, C++, Julia, \LaTeX, Matlab, Python
	\end{itemize}
\section{EDUCATION:}
	Rensselaer Polytechnic Institute, Troy, NY \\
	Bachelor's: Computer Science and Mechanical Engineering, May 2015 \\
	G.P.A. 3.95/4.00, Dean's List
 \section{EXPERIENCE:}      
  \experience{Research Internship:}{Quadrotors, Nathan Michael, CMU}
    {May 2014 -- August 2014}
  {
    \item NSF Research Experience for Undergraduates (REU)
    \item Implemented minimum snap, collision free, multi-vehicle trajectory
      generation
    \item Developed controller for tracking of discretized trajectories
    \item Developed system for persistent flight, recharging, and coverage of waypoints
    \item Videos:
      \begin{itemize}
        \item Interview: \url{https://www.youtube.com/watch?v=qqpLp2Ayqcw}
        \item Demo flight: \url{https://www.youtube.com/watch?v=8GPXf5lc8iY}
      \end{itemize}
    \item C++, Matlab, Python, ROS
  }
  \experience{Independent Study:}{Robotic Catching, Jeff Trinkle, RPI}{ August
  2013 -- December 2013}
  {
    \item Initial modeling and simulation of contact-oriented catching of a sliding object
    \item Matlab, C++
  }
  \experience{Research Internship:}{Wing Assembly, Reid Simmons, CMU}{May 2013 -- August 2013}
  {
    \item Developed a simulation of multi-robot assembly of an airplane wing-ladder
    \item Implemented an autonomous behavior for a mobile robot to attach and
      align to an airplane wing spar
    \item C++, Task Description Language
  }
  \experience{Research: }{High Performance Computing, SCOREC, RPI}{September 2012 -- December 2013}
  {
		\item Implemented threaded mesh I/O for the Parallel Unstructured Mesh Interface
    \item Transitioned software from partition-model-per-part to partition-model-per-process
		\item Debugged and tested FMDB/PUMI on the BlueGene/Q and Intel Phi
    \item C++, MPI
  }
  \experience{Undergraduate TA: }{Computer Science 1, RPI}{August 2012 -- May 2013}
  {
		\item Assisted students with lab-work and graded results
    \item Python
  }
  \experience{Research: }{Cognitive Robotics Lab, RPI}{September 2012 -- December 2012}
  {
		\item Development of server for Robot Operating System based actor
			controlled by Unity tablet application.
    \item Python
  }
\pagebreak[2]
  \experience{Work Study:}{Campus Computer Store, RPI}{September 2011 -- May 2012}
  {
   	\item Performed sales, managed stock, ensured placement of orders, organized items for
					inventory and display, delivered orders, and assisted in opening and closing
  }
\pagebreak[1]
\normalsize{\section{Presentations \& Publications}}
\nocite{persistentpaper}
\nocite{persistentposter}
\nocite{boeingposter}
\nocite{boeingpaper}
\renewcommand\refname{}
\bibliographystyle{IEEEtran}
\bibliography{bibliography}

\normalsize{\section{Hardware Projects}}
\begin{itemize}
  \item Developed a heart-rate monitor which turns off electronics when user
    falls asleep (Introduction to Engineering Design)
  \item Designed and produced a longboard press and several longboards using NX CAD, including a longboard
        with LED underlighting
  \item Designed and produced various themed CO2 cars including a racing pickup, Tron Light-Cycle,
        and autonomous golf cart, utilizing Solidworks, 3d printing, and CNC milling
\end{itemize}
\normalsize{\section{Software Projects}}
\begin{itemize}
  \item MultiQuadLift: linearized simulation of large swarms of quadrotors with
    relative measurements and payload carrying in Julia,
    \url{https://github.com/mcorah/MultiQuadLift}
  \item BeepHive: parallel robotic swarm simulator in C++ with MPI (WIP),
    \url{https://github.com/BeepBoopBop/BeepHive}
\end{itemize}
\section{ACTIVITIES:}
	\begin{itemize}
    \item RPI Sample Return Robot Challenge Team
    \begin{itemize}
      \item Competition involves autonomously finding and retrieving objects in an outdoor environment
      \item Lead mechanical and electrical design for the Summer 2014 competition
    \end{itemize}
    \item Upsilon Pi Epsilon (computer science honor society)
	\end{itemize}
\end{resume} 
\end{document}
