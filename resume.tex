% LaTeX file for resume
% This file uses the resume document class (res.cls)
%\documentclass[margin,letterpaper]{res}
\documentclass[margin]{res}
\usepackage{hyperref}
\usepackage{enumitem}

\topmargin=-.5in  % start text higher on the page
\footskip=-0.5in
\hoffset=0.5in
\textwidth=6.5in
\setlength{\textheight}{10.2in} % increase text height to fit resume on 1 page

\newcommand{\experience}[4]{
	\begin{tabular}{p{4in} r}
    {\bf #1} #2 & #3
  \end{tabular}
  \begin{itemize}
      #4
  \end{itemize}
}

\begin{document}
\name{Micah David Corah}

\address{\bf Permanent\\ 3361 E. Maplewood Ave. \\   Centennial, CO 80121\\ {\bf phone: }(720) 270-3901 \\ {\bf email: }micahcorah@gmail.com}
\address{\bf Current\\ 510 N. Negley Ave. \\ Apt. 2 \\   Pittsburgh, PA 15206  }

\begin{resume}
  \section{EDUCATION:}
  \begin{tabular}{ll}
    Carnegie Mellon University, Pittsburgh, PA & Rensselaer Polytechnic Institute, Troy, NY \\
    Doctorate: Robotics, Ongoing & Bachelor's: Computer Science \\
                                 &Bachelor's: Mechanical Engineering, May 2015 \\
                                 &G.P.A. 3.92/4.00
  \end{tabular}

 \section{EXPERIENCE:}
  \experience{Research Internship:}{Aerial Vehicles, Nathan Michael, CMU}
    {May 2014 -- August 2014}
  {
    \item NSF Research Experience for Undergraduates (REU)
    \item Implemented minimum snap, collision free, multi-vehicle trajectory
      generation
    \item Developed controller for tracking of discretized trajectories
    \item Developed system for persistent flight, recharging, and coverage of waypoints
    \item Interview: \url{https://www.youtube.com/watch?v=qqpLp2Ayqcw}
  }
  \experience{Independent Study:}{Robotic Catching, Jeff Trinkle, RPI}{ August
  2013 -- December 2013}
  {
    \item Initial modeling and simulation of contact-oriented catching of a sliding object
  }
  \experience{Research Internship:}{Wing Assembly, Reid Simmons, CMU}{May 2013 -- August 2013}
  {
    \item Developed a simulation of multi-robot assembly of an airplane wing-ladder
    \item Implemented an autonomous behavior for a mobile robot to attach and
      align to an airplane wing spar
  }
  \experience{Undergraduate Research:}{Parallel Computing, SCOREC, RPI}{September 2012 -- December 2013}
  {
		\item Implemented threaded mesh I/O for the Parallel Unstructured Mesh Interface
    \item Transitioned software from partition-model-per-part to partition-model-per-process
  }
  \experience{Undergraduate TA:}{Computer Science 1, RPI}{August 2012 -- May 2013}
  {
		\item Assisted students with lab-work and graded results
  }
\section{SKILLS AND INTERESTS:}
  \begin{itemize}
    \item Active perception
    \item Multi-robot exploration
    \item Multi-robot systems
    \item Control and trajectory generation
    \item Parallel, high performance, and scientific computing (MPI)
    \item Languages: C, C++, Julia, \LaTeX, Matlab, Python
	\end{itemize}

\normalsize{\section{Hardware Projects}}
\begin{itemize}
  \item Developed a heart-rate monitor which turns off electronics when user
    falls asleep (Introduction to Engineering Design)
  \item Designed and produced a longboard press and several longboards using NX CAD, including a longboard
        with LED underlighting
  \item Designed and produced various themed CO2 cars including a racing pickup, Tron Light-Cycle,
        and autonomous golf cart, utilizing Solidworks, 3d printing, and CNC milling
\end{itemize}
\normalsize{\section{Software Projects}}
\begin{itemize}
  \item MultiQuadLift: linearized simulation of large swarms of quadrotors with
    relative measurements and payload carrying in Julia,
    \url{https://github.com/mcorah/MultiQuadLift}
  \item BeepHive: parallel robotic swarm simulator in C++ with MPI (WIP),\\
    \url{https://github.com/BeepBoopBop/BeepHive}
\end{itemize}
\normalsize{\section{Presentations \& Publications}}
\nocite{persistentpaper}
\nocite{persistentposter}
\nocite{boeingposter}
\nocite{boeingpaper}
%\renewcommand\refname{}
\bibliographystyle{IEEEtran}
{
  \let\oldbibitem\bibitem
  \renewcommand{\bibitem}[1]{\itemindent5mm\oldbibitem {#1}}
  \renewcommand{\section}[2]{}
  \bibliography{bibliography}
}
\end{resume}
\end{document}
