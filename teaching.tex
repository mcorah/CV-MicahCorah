\cvsection{Teaching}

\begin{cventries}
  \cventry
  {Introduction to Artificial Intelligence (CSCI 404)}
  {Mines}
  {Instructor}
  {Spring 2024, 2025}
  {
    \begin{cvitems}
      \item Senior level undergraduate computer science course
      \item Textbook: \emph{Artificial Intelligence a Modern Approach} by
        Stuart Russell and Peter Norvig
    \end{cvitems}
  }
  \cventry
  {Special Topics: Planning for Perception}
  {Mines}
  {Instructor}
  {Fall 2024, 2025}
  {
    \begin{cvitems}
      \item Graduate level computer science course
      \item Course spanning active perception for robots, informative path
        planning, and submodular optimization applied to these topics
      \item Note: formerly titled \emph{Autonomous Sensins \& Perception}
    \end{cvitems}
  }
  \cventry
  {Robot Mobility on Air, Land, \& Sea (16-665)}
  {CMU}
  {Co-Instructor}
  {Fall 2022}
  {
    \begin{cvitems}
      \item Core course in the \href{https://mrsd.ri.cmu.edu/}{Masters in
        Robotics Systems Development} (MRSD) program
      \item Gave two lectures of the \emph{Aerial Mobility} component:
        \emph{Model Predictive and Adaptive Control}
        and
        \emph{Trajectory Generation and Tracking}
      \item Revised and expanded material for each lecture.
        Improved emphasis on concrete applications and introduced discussion of
        autonomy system design and safe navigation with respect to flatness-basd
        trajectory generation
      \item Collaborated with TAs to port section project from Matlab to Python
    \end{cvitems}
  }
  \cventry
  {Mathematical Fundamentals for Robotics (16-811)}
  {CMU}
  {Teaching Assistant}
  {Aug--Dec 2017}
  {
    Instructor: \emph{Prof. Michael Erdmann}\linebreak
    \begin{cvitems} % Description(s) bullet points
    \item Course: Mathematical Fundamentals for Robotics (16-811)
    \item Responsibilities: grading assignments, holding office hours
    \item Prepared and gave a lecture on submodular maximization
    \end{cvitems}
  }
\end{cventries}

